\section{The Latent Class Analysis Model}

\subsection{The Classic LCA Model}
In this section we will first introduce the classic LCA model and extend this model to include covariate effects using a multinomial logistic regression parametrization. Suppose there are M polytomous manifest variables or items, $u_{1},...,u_{M}$ observed on $N$ individuals in a data sample. Let $\bm{u_{i}} = (u_{1i},...,u_{Mi})^{\intercal}$ denote the full response pattern of individual $i$ where $u_{mi}$ is the response to item $m$. The classic LCA model represented in first diagram of Figure 1 (Lazarsfled \& Henry, 1968; McCutcheon, 1987) assumes that an underlying latent variable $C$ exist, consisting of $K$ latent classes, such that $C_{i}=k; k=1,2,...,K$ if individual $i$ belongs to class $k$. Path diagrams depicted in this thesis follows the convention used in Mplus version 8 User Guide (Muth\'en \& Muth\'en, 1998-2017). Thus, the distribution of a full response pattern can be expressed as 

\begin{align}
P(\bm{U_{i}}) &= \sum_{k=1}^{K} \bigg[ P(C_{i}=k) \cdot P(u_{1i},u_{2i},..,u_{M}|C_{i}=k)\bigg] \\
&=\sum_{k=1}^{K} \bigg[ \pi_{k} \cdot P(u_{1i},u_{2i},..,u_{M}|C_{i}=k)\bigg]
\end{align}
The last term of equation 2.2 can be simplified by $conditional$ $independence$ which is assumed on the M items conditional on latent class membership. This assumption implies that all the associations shared among the observed items is strictly caused by the latent variable or another way of putting it, conditional on the latent variable, responses to all of the manifest variables are assumed to be statistically independent. Conditional independence is depicted in the first diagram of Figure 1. As shown, the items are only connected by the latent variable as indicated by the one-way directional path which signifies that the K items are only related through the latent variable. Thus, we have the following 
\begin{align}
P(\bm{U_{i}}) = \sum_{k=1}^{K} \bigg[ \pi_{k} \cdot  \prod_{m=1}^{M}\prod_{r=1}^{R_{m}} P(u_{mi}=r|c_{i}=k)^{\mathbbm{1}(u_{mi} = r)} \bigg ]
\end{align}
where $\mathbbm{1}(u_{mi} = r)=1$ if $u_{mi}=r$ and $0$ otherwise. The classic LCA model is defined by two sets of parameters: the item response and class proportion probabilities. The item response probabilities are the $measurement$ $parameters$ and describes the relationship between the latent variable and items observed (i.e. it indicates how likely an individual will endorse an item within each latent class and so these results are the basis for the overall interpretation of the latent classes). The vector of item response probabilities sum to 1 conditional on latent class since an individual provides one and only one response to each item within each class. The class proportion probabilities are regarded as the $structural$ $parameters$ as it characterizes the overall distribution of the latent variable, which in the class LCA model specify the relative size of each latent class. Similarly, latent classes are mutually exclusive and exhaustive implying that individuals can only belong to one and only one latent class, hence, the class probabilities sum to 1. Three common parametrization are commonly used that all result in same LCA models: the probability parameterization, the log-linear parameterization and the multinational logistic regression parameterization. For this thesis, the distribution of latent variable is parameterized with a standard multinomial logistic regression. The equation is
\begin{align}
    \pi_{k} = P(C=k) = \frac{exp(\beta_{k})}{1 + \sum_{h=1}^{K-1}exp(\beta_{h})}
\end{align} where the reference category is latent class K such that $\beta_{K}=0$. The choice of the reference category is arbitrary and will not affect the results, however, it can impact the ease of interpretation.

The distribution of the item-response probabilities may also be parameterized under a logistic regression framework. However, for the purpose of this thesis, a different modelling framework is specified, known as the ordinal latent response formulation. It may seem odd to assume items are ordinal categorical variables, however, in several substantive fields, a natural ranking exists among the responses categories. For example attitudinal questions on social and public opinion surveys follow a Likert-type scale that range from "strongly disagree" to "strongly agree" or from "least important" to "most important." Responses to other survey questions such as "never," "sometimes", "often", or "always" are also examples of ordered categories (Liao, 1994). Additionally, this is also the chosen parameterization implemented in Mplus Software (Muth\'en \& Muth\'en, 2018). In the following, we will examine the ordinal latent response formulation in further detail. The following discussion follows from Skrondal \& Hesketh (2004) and Masyn (2017). The item $u_{mi}$, $m=1,2,...,M$ is an ordinal variable with $R$, possible response categories and can be viewed as a discretized form of a $continuous$ $latent$ $response$ $variable$ $u_{mi}^{*}$ or another way of putting it $u_{mi}^{*}$ an underlying continuum of $u_{mi}$. Assume that the relationship between $u_{mi}$ and $u_{mi}^{*}$ is modelled by the following, 
\begin{align}
    u_{mi}^{*} = \bm{x}_{i}^\intercal \bm{\beta} + \epsilon_{i}
\end{align} where $\bm{x}$ is a vector specifying covariate effects, $\bm{\beta}$ are the corresponding regression parameters and $\epsilon_{i}$ is the random component that follows a standard logistic distribution. However, note that the classic LCA model assume no covariate effects $\bm{x}^{\intercal} = \bm{0}^{\intercal}$. LCA models with covariate effects will be discussed in the following subsections. Now, if $u_{mi}$ was a continuous variable then $u_{mi} = u_{mi}^{*}$. For the case of a binary ordinal variable, the relationship between $u_{mi}$ and $u_{mi}^{*}$ conditional on latent class k is given by
\begin{align}
    u_{mi}=
    \begin{cases}
    \hspace{1em}0 & \text{if } $ -\infty < u_{mi}^{*} \leq \tau_{1k}$ \\
    \hspace{1em}1 & \text{if } $\tau_{1k} < u_{mi}^{*} \leq \infty $
    \end{cases}
\end{align} such that $\tau_{1k}$ is the cutoff point or threshold value that separate the response categories. For all individuals in population with $u_{mi}^{*}$ values less than $\tau_{1k}$ will manifest the outcome variable $u_{mi}$=0 and those greater than $\tau_{1k}$ will manifest $u_{mi}=1$. Based on the relations above, the item-response probabilities for a binary item is defined as
\begin{align}
    P(u_{mi}=1 | c=k) &= P(\tau_{1k} \leq u_{mi}^{*}) \nonumber \\
    &= P(\tau_{1k} \leq \bm{x_{i}}^{\intercal}\bm{\beta} + \bm{\epsilon}_{i}) \nonumber \\
    &= P( \bm{\epsilon}_{i} \geq \tau_{1k}) \nonumber \\
    &= 1- F_{\epsilon}(\tau_{1k}) \nonumber 
\end{align} Since $\epsilon_{i}$ $\sim$ $Logistic(0,1)$ therefore we have that,
\begin{align}
P(u_{mi}=1|c=k) &= \frac{1}{1+exp(\tau_{1k})} 
\end{align} where $\tau_{1k}$ is the negative log odds of endorsing item $u_{mi}$ given membership in latent class, i.e. 
\begin{align}
     \tau_{1k} = -log\bigg( \frac{P(u_{mi=1}|c=k)}{P(u_{mi=0}|c=k)}\bigg)
\end{align}
For an item with $R$ response categories and thereby $(R-1)$ thresholds. The latent response formulation model for $R$ responses is therefore,
\begin{align}
    u_{mi} =
    \begin{cases}
    \hspace{1em}0 & \text{if } $-\infty < u_{mi}^{*} \leq \tau_{1k}$ \\
    \hspace{1em}1 & \text{if } $\tau_{1k} < u_{mi}^{*} \leq \tau_{2k}$ \\
    &\hspace{0.2em}\vdots \\
    R-1 & \text{if } $\tau_{(R-1)k} < u_{mi}^{*} \leq \infty$ 
    \end{cases}
\end{align} where $\tau_{(r-1)k}$ and $\tau_{rk}$ are the lower and upper thresholds respectively where each range corresponds to the $rth$ response category on the $u_{mi}$ scale. In this case, the item response probability is constructed based on cumulative probabilities as we are interested in obtaining the probability of attaining at least a given response probability, i.e. $P(u_{mi} \leq r) = P(u_{mi}^{*} \leq \tau_{r+1})$. Similarly to the binary case we have that,
\begin{align}
    P(u_{mi} \leq r | c=k) &= P(u_{mi}^{*} \leq \tau_{(r+1)k}) \nonumber \\
    &= P(\bm{x}^{\intercal}_{i}\bm{\beta} + \bm{\epsilon}_{i} \leq \tau_{(r+1)k}) \nonumber \\
    &= P(\bm{\epsilon}_{i} \leq \tau_{(r+1)k}) \nonumber \\
    &= \frac{1}{1+exp(-\tau_{(r+1)k})}
\end{align} Therefore, the individualized item-response probabilities for item $m$ in class $k$ with $R$ response categories are
\begin{align}
    P(u_{mi}=0|c=k) &= F(\tau_{m1}) \nonumber \\
    P(u_{mi}=1|c=k) &= F(\tau_{m2}) - F(\tau_{m1}) \nonumber \\ 
    &\vdots \nonumber \\
    P(u_{mi}=R|c=k) &= 1 - F(\tau_{R-1}) 
\end{align} where $F(\cdot)$ represent the standard logistic cumulative distribution 
function.

\subsection{Model Interpretation}





